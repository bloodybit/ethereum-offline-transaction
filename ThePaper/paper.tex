


\title{Identity Management in Blockchain}
\author{
        \textsc{Riccardo Sibani}
            \qquad
        \textsc{Filippo Boiani}
        \mbox{}\\
        \normalsize
            \texttt{rsibani@kth}
        \textbar{}
            \texttt{boiani@kth.se}
        % \normalsize
        %     \texttt{@kth.se}
}
\date{\today}

\documentclass[12pt,twoside]{article}

\usepackage[paper=a4paper,dvips,top=1.5cm,left=1.5cm,right=1.5cm,
    foot=1cm,bottom=1.5cm]{geometry}


%\usepackage[T1]{fontenc}
%%\usepackage{pslatex}
\renewcommand{\rmdefault}{ptm} 
\usepackage{mathptmx}
\usepackage[scaled=.90]{helvet}
\usepackage{courier}
%
\usepackage{bookmark}
\usepackage{newclude}

\usepackage{fancyhdr}
\providecommand{\keywords}[1]{\textbf{\textit{Index terms---}} #1}
\pagestyle{fancy}

\usepackage{rotating}
\newcommand\tabrotate[1]{\begin{turn}{90}\rlap{#1}\end{turn}}


 % Useful Basics
 \usepackage{color}
 \usepackage{framed}
 \usepackage{listings}

 % Write Answer
 \newenvironment{notation}{%
   \def\FrameCommand{\colorbox{yellow!20}}%
   \MakeFramed{\advance\hsize-\width \FrameRestore}}
 {\endMakeFramed}
 
%%----------------------------------------------------------------------------
%%   pcap2tex stuff
%%----------------------------------------------------------------------------
 \usepackage[dvipsnames*,svgnames]{xcolor} %% For extended colors
 \usepackage{tikz}
 \usetikzlibrary{arrows,decorations.pathmorphing,backgrounds,fit,positioning,calc,shapes}

%% \usepackage{pgfmath}	% --math engine
%%----------------------------------------------------------------------------
%% \usepackage[latin1]{inputenc}
\usepackage[utf8]{inputenc} % inputenc allows the user to input accented characters directly from the keyboard
\usepackage[swedish,english]{babel}
%% \usepackage{rotating}		 %% For text rotating
\usepackage{array}			 %% For table wrapping
\usepackage{graphicx}	                 %% Support for images
\usepackage{float}			 %% Suppor for more flexible floating box positioning
\usepackage{color}                       %% Support for colour 
\usepackage{mdwlist}
%% \usepackage{setspace}                 %% For fine-grained control over line spacing
%% \usepackage{listings}		 %% For source code listing
%% \usepackage{bytefield}                %% For packet drawings
\usepackage{tabularx}		         %% For simple table stretching
%%\usepackage{multirow}	                 %% Support for multirow colums in tables
\usepackage{dcolumn}	                 %% Support for decimal point alignment in tables
\usepackage{url}	                 %% Support for breaking URLs
\usepackage[perpage,para,symbol]{footmisc} %% use symbols to ``number'' footnotes and reset which symbol is used first on each page

%% \usepackage{pygmentize}           %% required to use minted -- see python-pygments - Pygments is a Syntax Highlighting Package written in Python
%% \usepackage{minted}		     %% For source code highlighting

%% \usepackage{hyperref}		
\usepackage[all]{hypcap}	 %% Prevents an issue related to hyperref and caption linking
%% setup hyperref to use the darkblue color on links
%% \hypersetup{colorlinks,breaklinks,
%%             linkcolor=darkblue,urlcolor=darkblue,
%%             anchorcolor=darkblue,citecolor=darkblue}

%% Some definitions of used colors
\definecolor{darkblue}{rgb}{0.0,0.0,0.3} %% define a color called darkblue
\definecolor{darkred}{rgb}{0.4,0.0,0.0}
\definecolor{red}{rgb}{0.7,0.0,0.0}
\definecolor{lightgrey}{rgb}{0.8,0.8,0.8} 
\definecolor{grey}{rgb}{0.6,0.6,0.6}
\definecolor{darkgrey}{rgb}{0.4,0.4,0.4}
%% Reduce hyphenation as much as possible
\hyphenpenalty=15000 
\tolerance=1000

%% useful redefinitions to use with tables
\newcommand{\rr}{\raggedright} %% raggedright command redefinition
\newcommand{\rl}{\raggedleft} %% raggedleft command redefinition
\newcommand{\tn}{\tabularnewline} %% tabularnewline command redefinition

%% definition of new command for bytefield package
\newcommand{\colorbitbox}[3]{%
	\rlap{\bitbox{#2}{\color{#1}\rule{\width}{\height}}}%
	\bitbox{#2}{#3}}

%% command to ease switching to red color text
\newcommand{\red}{\color{red}}
%%redefinition of paragraph command to insert a breakline after it
\makeatletter
\renewcommand\paragraph{\@startsection{paragraph}{4}{\z@}%
  {-3.25ex\@plus -1ex \@minus -.2ex}%
  {1.5ex \@plus .2ex}%
  {\normalfont\normalsize\bfseries}}
\makeatother

%%redefinition of subparagraph command to insert a breakline after it
\makeatletter
\renewcommand\subparagraph{\@startsection{subparagraph}{5}{\z@}%
  {-3.25ex\@plus -1ex \@minus -.2ex}%
  {1.5ex \@plus .2ex}%
  {\normalfont\normalsize\bfseries}}
\makeatother

\setcounter{tocdepth}{3}	%% 3 depth levels in TOC
\setcounter{secnumdepth}{5}
%%%%%%%%%%%%%%%%%%%%%%%%%%%%%%%%%%%%%%%%%%%%%%%%%%%%%%%%%%%%%%%%%%%%
%% End of preamble
%%%%%%%%%%%%%%%%%%%%%%%%%%%%%%%%%%%%%%%%%%%%%%%%%%%%%%%%%%%%%%%%%%%%

\renewcommand{\headrulewidth}{0pt}
\lhead{II2202, Fall 2017, Period 1-2}
%% or \lhead{II2202, Fall 2016, Period 1}
\chead{Draft project report}
\rhead{\date{\today}}

\makeatletter
\let\ps@plain\ps@fancy 
\makeatother

\setlength{\headheight}{15pt}
\begin{document}

\maketitle

\begin{notation}
    To finish and correct once the report is done!
\end{notation}

\begin{abstract}
\label{sec:abstract}

%\begin{abstract}
%% Text of abstract
Suspendisse potenti. Suspendisse quis sem elit, et mattis nisl. Phasellus consequat erat eu velit rhoncus non pharetra neque auctor. Phasellus eu lacus quam. Ut ipsum dolor, euismod aliquam congue sed, lobortis et orci. Mauris eget velit id arcu ultricies auctor in eget dolor. Pellentesque suscipit adipiscing sem, imperdiet laoreet dolor elementum ut. Mauris condimentum est sed velit lacinia placerat. Vestibulum ante ipsum primis in faucibus orci luctus et ultrices posuere cubilia Curae; Nullam diam metus, pharetra vitae euismod sed, placerat ultrices eros. Aliquam tincidunt dapibus venenatis. In interdum tellus nec justo accumsan aliquam. Nulla sit amet massa augue.
\end{abstract}

\begin{keyword}
Science \sep Publication \sep Complicated
%% keywords here, in the form: keyword \sep keyword

%% MSC codes here, in the form: \MSC code \sep code
%% or \MSC[2008] code \sep code (2000 is the default)

\end{keyword}

\end{frontmatter}     % abstract and keywords


%% Text of abstract
SONIC is an identity management protocol that allows to have a unique user social profile and to communicate with other users on different online social networks (OSN). Sonic relies on a distributed system called Global Social Lookup System. GSLS can be seen as a service mapping a user global identity to its social profiles. Is open source and based on a distributed hash table (DHT) to handle communication and persistence. This implementation and its openness carries a list of security issue. 

The aim of this project is to overcome these issues by exploiting the characteristic of blockchain. In fact, consensus, distribution and unforgeability of blockchain are valuable features in a context where information needs to be distributed and everybody needs to agree on the same state. 
 
Hence, we describe some of the security issues related to the current implementation of GSLS and propose a new design that takes advantage of blockchain. The new system is then evaluated in terms of performance, security, privacy and invasiveness for the user. 

The results show that the performances of the read operations are comparable to the current implementation while the cost of the write operations are bound to the public blockchain mining time. Therefore there is a tradeoff between security and performances depending on the limitations to the number of transactions that can be processed per second. 
\\

\keywords{Blockchain, Identity Management, Distributed Systems, OSNs}
%% keywords here, in the form: keyword \sep keyword

%% MSC codes here, in the form: \MSC code \sep code
%% or \MSC[2008] code \sep code (2000 is the default)



\end{abstract}

\clearpage

\selectlanguage{english}
\tableofcontents

\section*{List of Acronyms and Abbreviations}
\label{list-of-acronyms-and-abbreviations}

This document requires readers to be familiar with terms and concepts of blockchain technology, identity management and distributed networks. For clarity we summarize some of these terms and give a short description of them before presenting them in next sections.

\begin{basedescript}{\desclabelstyle{\pushlabel}\desclabelwidth{10em}}
\item[GSLS Client]              It is a piece of code that allows the user to interact and send request to the GSLS server. In our last approach, the client is implemented as a Desktop application.
\item[GSLS server (or node)]    The GSLS server is a piece of code that exposes api callable by the GSLS client via HTTP protocol. The server is also a blockchain client in that it is either connected to an Ethereum node or it is an actual node in the network.
\item[DHT]                      A distributed hash table (DHT) is a class of a decentralised distributed system that provides a lookup service similar to a hash table.
\item[SR]                       Social record, the user is the one that owns a social record and has the right to create and update it as well as to search for other social records. The user does this by interacting with an application, the GSLS client.
\end{basedescript}


\clearpage

\include*{sections/02_introduction} % introduction to the topic 
\include*{sections/03_related-works} % theoretical framework and literature study 
\include*{sections/04_hypotheses} % hypotheses, premises and considerations
\include*{sections/05_methods}      % methods used to validate our assumptions
\include*{sections/06_results}   % result and analysis 
\include*{sections/07_discussion}   % discussion



\bibliographystyle{IEEEtran}
%%\bibliographystyle{myIEEEtran}

\bibliography{bibliography/bib}
\appendix
% \section{Insensible Approximation}

% Note that the Appendix or Appendices are Optional.


\end{document}