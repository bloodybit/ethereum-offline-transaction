


\title{Identity Management in Blockchain}
\author{
        \textsc{Riccardo Sibani}
            \qquad
        \textsc{Filippo Boiani}
        \mbox{}\\
        \normalsize
            \texttt{rsibani@kth}
        \textbar{}
            \texttt{boiani@kth.se}
        % \normalsize
        %     \texttt{@kth.se}
}
\date{\today}

\documentclass[12pt,twoside]{article}

\usepackage[paper=a4paper,dvips,top=1.5cm,left=1.5cm,right=1.5cm,
    foot=1cm,bottom=1.5cm]{geometry}


%\usepackage[T1]{fontenc}
%%\usepackage{pslatex}
\renewcommand{\rmdefault}{ptm} 
\usepackage{mathptmx}
\usepackage[scaled=.90]{helvet}
\usepackage{courier}
%
\usepackage{bookmark}

\usepackage{fancyhdr}
\providecommand{\keywords}[1]{\textbf{\textit{Index terms---}} #1}
\pagestyle{fancy}

\usepackage{rotating}
\newcommand\tabrotate[1]{\begin{turn}{90}\rlap{#1}\end{turn}}


 % Useful Basics
 \usepackage{color}
 \usepackage{framed}
 \usepackage{listings}

 % Write Answer
 \newenvironment{notation}{%
   \def\FrameCommand{\colorbox{yellow!20}}%
   \MakeFramed{\advance\hsize-\width \FrameRestore}}
 {\endMakeFramed}
 
%%----------------------------------------------------------------------------
%%   pcap2tex stuff
%%----------------------------------------------------------------------------
 \usepackage[dvipsnames*,svgnames]{xcolor} %% For extended colors
 \usepackage{tikz}
 \usetikzlibrary{arrows,decorations.pathmorphing,backgrounds,fit,positioning,calc,shapes}

%% \usepackage{pgfmath}	% --math engine
%%----------------------------------------------------------------------------
%% \usepackage[latin1]{inputenc}
\usepackage[utf8]{inputenc} % inputenc allows the user to input accented characters directly from the keyboard
\usepackage[swedish,english]{babel}
%% \usepackage{rotating}		 %% For text rotating
\usepackage{array}			 %% For table wrapping
\usepackage{graphicx}	                 %% Support for images
\usepackage{float}			 %% Suppor for more flexible floating box positioning
\usepackage{color}                       %% Support for colour 
\usepackage{mdwlist}
%% \usepackage{setspace}                 %% For fine-grained control over line spacing
%% \usepackage{listings}		 %% For source code listing
%% \usepackage{bytefield}                %% For packet drawings
\usepackage{tabularx}		         %% For simple table stretching
%%\usepackage{multirow}	                 %% Support for multirow colums in tables
\usepackage{dcolumn}	                 %% Support for decimal point alignment in tables
\usepackage{url}	                 %% Support for breaking URLs
\usepackage[perpage,para,symbol]{footmisc} %% use symbols to ``number'' footnotes and reset which symbol is used first on each page

%% \usepackage{pygmentize}           %% required to use minted -- see python-pygments - Pygments is a Syntax Highlighting Package written in Python
%% \usepackage{minted}		     %% For source code highlighting

%% \usepackage{hyperref}		
\usepackage[all]{hypcap}	 %% Prevents an issue related to hyperref and caption linking
%% setup hyperref to use the darkblue color on links
%% \hypersetup{colorlinks,breaklinks,
%%             linkcolor=darkblue,urlcolor=darkblue,
%%             anchorcolor=darkblue,citecolor=darkblue}

%% Some definitions of used colors
\definecolor{darkblue}{rgb}{0.0,0.0,0.3} %% define a color called darkblue
\definecolor{darkred}{rgb}{0.4,0.0,0.0}
\definecolor{red}{rgb}{0.7,0.0,0.0}
\definecolor{lightgrey}{rgb}{0.8,0.8,0.8} 
\definecolor{grey}{rgb}{0.6,0.6,0.6}
\definecolor{darkgrey}{rgb}{0.4,0.4,0.4}
%% Reduce hyphenation as much as possible
\hyphenpenalty=15000 
\tolerance=1000

%% useful redefinitions to use with tables
\newcommand{\rr}{\raggedright} %% raggedright command redefinition
\newcommand{\rl}{\raggedleft} %% raggedleft command redefinition
\newcommand{\tn}{\tabularnewline} %% tabularnewline command redefinition

%% definition of new command for bytefield package
\newcommand{\colorbitbox}[3]{%
	\rlap{\bitbox{#2}{\color{#1}\rule{\width}{\height}}}%
	\bitbox{#2}{#3}}

%% command to ease switching to red color text
\newcommand{\red}{\color{red}}
%%redefinition of paragraph command to insert a breakline after it
\makeatletter
\renewcommand\paragraph{\@startsection{paragraph}{4}{\z@}%
  {-3.25ex\@plus -1ex \@minus -.2ex}%
  {1.5ex \@plus .2ex}%
  {\normalfont\normalsize\bfseries}}
\makeatother

%%redefinition of subparagraph command to insert a breakline after it
\makeatletter
\renewcommand\subparagraph{\@startsection{subparagraph}{5}{\z@}%
  {-3.25ex\@plus -1ex \@minus -.2ex}%
  {1.5ex \@plus .2ex}%
  {\normalfont\normalsize\bfseries}}
\makeatother

\setcounter{tocdepth}{3}	%% 3 depth levels in TOC
\setcounter{secnumdepth}{5}
%%%%%%%%%%%%%%%%%%%%%%%%%%%%%%%%%%%%%%%%%%%%%%%%%%%%%%%%%%%%%%%%%%%%
%% End of preamble
%%%%%%%%%%%%%%%%%%%%%%%%%%%%%%%%%%%%%%%%%%%%%%%%%%%%%%%%%%%%%%%%%%%%

\renewcommand{\headrulewidth}{0pt}
\lhead{II2202, Fall 2017, Period 1-2}
%% or \lhead{II2202, Fall 2016, Period 1}
\chead{Draft project report}
\rhead{\date{\today}}

\makeatletter
\let\ps@plain\ps@fancy 
\makeatother

\setlength{\headheight}{15pt}
\begin{document}

\maketitle

\begin{notation}
    To finish and correct once the report is done!
\end{notation}

\begin{abstract}
\label{sec:abstract}

%\begin{abstract}
%% Text of abstract
Suspendisse potenti. Suspendisse quis sem elit, et mattis nisl. Phasellus consequat erat eu velit rhoncus non pharetra neque auctor. Phasellus eu lacus quam. Ut ipsum dolor, euismod aliquam congue sed, lobortis et orci. Mauris eget velit id arcu ultricies auctor in eget dolor. Pellentesque suscipit adipiscing sem, imperdiet laoreet dolor elementum ut. Mauris condimentum est sed velit lacinia placerat. Vestibulum ante ipsum primis in faucibus orci luctus et ultrices posuere cubilia Curae; Nullam diam metus, pharetra vitae euismod sed, placerat ultrices eros. Aliquam tincidunt dapibus venenatis. In interdum tellus nec justo accumsan aliquam. Nulla sit amet massa augue.
\end{abstract}

\begin{keyword}
Science \sep Publication \sep Complicated
%% keywords here, in the form: keyword \sep keyword

%% MSC codes here, in the form: \MSC code \sep code
%% or \MSC[2008] code \sep code (2000 is the default)

\end{keyword}

\end{frontmatter}     % abstract and keywords


%% Text of abstract
SONIC is an identity management protocol that allows to have a unique user social profile and to migrate it between different online social networks (OSN). Sonic relies on a distributed system called Global Social Lookup System. GSLS can be seen as a service mapping a user global identity to its social profiles. Is open source and based on a distributed hash table (DHT) to handle communication and persistence. This implementation and its openness carries a list of security issue. The aim of this project is to overcome these issues by exploiting the characteristic of blockchain. 

Consensus, distribution and unforgeability of blockchain are valuable characteristics in a context where information needs to be shared and everybody needs to agree on the same state. 
 
Hence, we describe some of the security issues related to the current implementation of GSLS and propose a new design that takes advantage of blockchain. The new system is then evaluated in terms of performance, security, privacy and invasiveness for the user. 

The results show that the performances of the read operations are comparable to current implementation while the cost of the write operations are bound to the public blockchain mining time that limits the maximum number of transactions that can be processed per second. Therefore there is a tradeoff between security and performances.

\keywords{
Blockchain, Identity Management, Distributed Systems,  OSNs}
%% keywords here, in the form: keyword \sep keyword

%% MSC codes here, in the form: \MSC code \sep code
%% or \MSC[2008] code \sep code (2000 is the default)



\end{abstract}

\clearpage

\selectlanguage{english}
\tableofcontents

\section*{List of Acronyms and Abbreviations}
\label{list-of-acronyms-and-abbreviations}

This document requires readers to be familiar with terms and concepts of blockchain technology, identity management and distributed networks. For clarity we summarize some of these terms and give a short description of them before presenting them in next sections.

\begin{basedescript}{\desclabelstyle{\pushlabel}\desclabelwidth{10em}}
\item[GSLS Client]              It is a piece of code that allows the user to interact and send request to the GSLS server. In our last approach, the client is implemented as a Desktop application.
\item[GSLS server (or node)]    The GSLS server is a piece of code that exposes api callable by the GSLS client via HTTP protocol. The server is also a blockchain client in that it is either connected to an Ethereum node or it is an actual node in the network.
\item[DHT]                      A distributed hash table (DHT) is a class of a decentralised distributed system that provides a lookup service similar to a hash table.
\item[SR]                       Social record, the user is the one that owns a social record and has the right to create and update it as well as to search for other social records. The user does this by interacting with an application, the GSLS client.
\end{basedescript}


\clearpage



\section{Introduction}
\label{S:1}

We are currently living in an era where applications are heavily data-centric and rely on information provided by users. It is the era of the \textit{Online Social Networks} (OSN) \cite{gondor_sonic:_2014} (e.g. Facebook, Twitter), where a person is analyzed in terms of interests and interaction with peers, building a network of connections and messages. As time passes, this information converges to create a digital identity. However, this identity is bound to a particular application which has the power to influence the behavior of the person, how she interacts and what she sees. This is a well-calculated lock-in effect used to bind users to a service \cite{gondor_distributed_2016}. 
\\

Result of the aforementioned model is that a person tends to segregate identities (social profiles) with replicated information in different OSN platforms. The platforms cannot communicate to each other without plugins or services. SONIC proposes an approach to overcome these problem by creating a protocol to enable communication and migration of user social profiles from one platform to another \cite{gondor_sonic:_2014}. The user profile is not replicated but identified by a global id that is not connected to any particular application. The management of user data in unified \cite{identity_mgmt_2003}, giving the possibility to access the same unique piece of information from any social network. This can, in turn, improve the privacy of the user data
%(e.g. interact with people even if not in the same OSN).
, since the OSN knows nothing about the user except its global id.
\\

To resolve identifiers to the actual user profile, SONIC takes advantage of a distributed system called \textit{Global Social Lookup System} (GSLS) \cite{gondor_distributed_2016}. User's \textit{GobalID} and profile location are stored in an dataset, called \textit{Social Record}, managed by the GSLS \cite{gondor_distributed_2016}.
\\

GSLS is currently implemented with Java and exposes APIs for creating, updating and querying the Social Records. The Social Records are stored in a Distributed Hash Table \cite{tomp2p_2009} and validated with RSA private-public key encryption. However, this open-source design has some security issues. As an example, we describe one attack: it is possible to download the source code, spawn an instance of the service and add it to the network of GSLS nodes. The malicious instance can take an outdated but signed Social Record and send it to the network, overriding the current entry. This can be achieved because data is checked against the P2P node's signature but nor against a timestamp nor a nonce. Therefore, is not possible to infer if the entry is the most recent one. 
\\

To overcome this and other security issues, we propose a new design that replace the DHT with blockchain \cite{nakamoto_bitcoin_2008}, taking advantage of distributed consensus, Byzantine fault tolerance, encryption, and unforgeability properties. The system employs Ethereum as a blockchain implementation \cite{wood_ethereum_2014}. The reason is based on the fact that Ethereum provides users with sufficient storage and a Turing-complete programming language that runs on a virtual machine, called Ethereum Virtual Machine (EVM) \cite{wood_ethereum_2014}. Therefore, it is possible to use the blockchain to handle logic and execute functions. In addition, can be used both as a validation (e.g. storing the hash of an entry) and storage system. 
\\

Each instance of the GSLS system exposes a RESTful API that holds an Ethereum node and interacts with the blockchain network. For the sake of our discussion, we will use the terms Ethereum and blockchain as synonyms.
\\

The remainder of this report is structured as follows: the next section presents an overview of related works, technical framework, and background. Section 3 and 4 provide a specification of our research questions and the description of possible designs to solve our research problem. Section 5 provides an analysis of the design in terms of security and invasiveness from the user perspective, as well as an evaluation of the performances of the implementation. In the last section we show the pro and cons of the proposed solution and then we conclude the report with a description of future work. 
 % introduction to the topic 
\section{Theoretical Framework and Literature Study}
\label{S:2}

Some text here.

\subsection{The Sonic project}

The way people enjoy the news and interact within their network and friends changed with the advent of the OSN (Online Social network) platforms. \par
Nowadays, OSNs are the main communication media but the information they generate are proprietary and, because of it, it is easy to imagine the reason behind why they try to create the so called \textit{lock in} effect.
In fact, the information about the users generated by surfing on the platform, once collected, allows the OSN to create personalized advertisements. Such behaviour, incentivises the social network platforms to implement barriers to exit. \par
What SONIC aims to reach is to connect all the platforms in a ``decentralized and ethereogeneus federation of OSN platforms'' (http://sonic-project.net/), via a specific protocol which allows to migrate and interact with people registered among different OSNs \cite{gondor_sonic:_2014}. \par
User identification is performed via a generated Unified Unique Identifier (UUID) derived from a \textit{PKCS\#8} \cite{pkcs8}with 8 random bytes added at the end, which is then decrypted with \textit{PBKDF\#2} \cite{pkcs8} with settings SHA256 \cite{hansen_us} for 10 000 iterations.
The output is finally a 256bit long field which is converted in base36, generating the definitive UUID. \par

Example of a UUID is: \\ \textbf{3R2IWN230NFI2QBYUDEQW02134DBSUIBPPOFWCDIN221343EEE} \par

Accounts are stored in a distributed directory service named \textit{Global Social Lookup System} which use a DHT in order to store and retrieve the single accounts. In particular, the accounts are hashed and based on the hash stored in a subset of GSLS Servers (consistent hashing). \par

Each user has 2 asymmetric keys: \textit{PersonalKeyPair} and \textit{AccountKeyPair}. The former to sign the registration and update the account information in the DTH, the latter signs and verify every payload exchange through the SONIC implementation.

\subsection{Blockchain}

DHT as a storage layer seemed to be the best solution in the previous implementation, but unfortunately the adopted framework TOMP2P \cite{tomp2p:2017} does not provide any functionality to validate the peer transmissions. 
This lack results in the possibility to send malicious information (such as delete users).
Thus, given the open source distribution licence, the risk of such attack is higher since anyone can set up its own GSLS node and join the network: compromising the entire system. \par

Another secondary problem is the distributed nature of the DHT which, in the current system, distributes the records with indirect replication \cite{_tomp2p_2017} (a certain number \textit{n} of nodes store the same social record). Nevertheless the users' information must be quickly retrievable by the main OSNs and in the current solution there may be some delays. \par

One last issue of the current GSLS is that the records are store in the memory (RAM) and in case of failure or switch off of the GSLS server, all the records on that server will be lost. \par

\subsection{Requirements of the solution}
The features that the new implementation requires:
\begin{itemize}
  \item Consistent and distributed storage, able to eventually converge to the same state and to persist in case of failures.
  \item A secure procedure to create, update and delete the social records of the users. The new solution must, in addition, be able to guarantee the security of the database even in case of a malicious attack or faulty behavior of the peer nodes.
  \item The new solution should be able to guarantee a faster lookup of the records based on the UUID key. This will allow to quickly provide the user's information to the requesting OSN or user. It may seem a secondary issue but, in a live environment, in order to convince the major OSNs to join such system, the solution should not reflect negatively on the performance.
\end{itemize}

To summarize, the project aims to find a solution able to guarantee security and agreement in an open source environment, where different competitors share the same data structure (meaning the social record). \par











% Short problem, 
% Hypothesis,
% a bit of what we will do and what we expect as outcome,

% The blockchain technology, born in 2009 from \textbf{Satoshi Nakamoto} [check if it is spelled correctly and place a reference to the paper], is still in its embryonal phase and its possibilities are yet not entirely explored.

% The main feature of this distributed ledger is to eventually reach an unanimous agreement among the different ledgers guaranteeing the integrity of the data. 

% Scalability is another key feature of the blockchain, each and every ledger act like a node in a distributed system: meaning that an application running on a server can easily access the local node with almost no latency. \textbf{<- Latency? }
% \\
% Given these premises, blockchain seems to be the right tool for the job: providing a consistent, distribute and scalable and environment, accessible by all the player at the same time.






% \subsubsection{Sonic}
% \subsubsection{GSLS}


% \subsection{Directory service}



% \subsection{DHT with P2P networks}

% \subsubsection{DHT}

% \subsubsection{Encription over the communication channels}




% \subsection{Blockchain}


% \subsubsection{Transaction centric}
% \subsubsection{State centric}



% \subsection{Blockchains on the market}

% \subsubsection{Bitcoin}

% \subsubsection{Ethereum}

% \subsubsection{Hyperledger - Fabric}

% \subsubsection{BigchainDb}
% [No bft]



% \subsection{Fruition of the blockchain}

% \subsubsection{Hot Wallet}
% \subsubsection{Cold Wallet}

% \subsubsection{Light clients}

% \subsection{Blockstack as an example of blockchain utilization in Open Source enviroment}
% Blockstack is a particular implementation of a decentralized DNS system based on blockchain. It combines DNS functionality with public key infrastructure and is primarily meant to be used by new blockchain applications.

% According to the company: \enquote{under the hood, Blockstack provides a decentralized domain name system (DNS), decentralized public key distribution system, and registry for apps and user identities} \cite{BlockStackMainPage}.

% The real breakthrough is the architecture the system is built on. It can be described as a three-layer design with the blockchain as the first and lower tier, the storage system as the upper and the peer network as middle layer.

% \subsection{Encryption}
% \subsubsection{Elliptic Curve}
% secp256k1
% https://crypto.stackexchange.com/questions/18965/is-secp256r1-more-secure-than-secp256k1

% \subsubsection{Asymmetric Encription <-- Needed????}



 % theoretical framework and literature study 

\section{Design GSLS 2.0}

\subsection{Choosing the blockchain}
The choice of the most appropriate blockchain technologyshould take into account:

\begin{itemize}
  \item Acknowledged security: the blockchain should not contain any technical flaw and be recognized by the community as \textit{trustable}. This is important since many projects are based on a good and promising concept but often times fail on the practical implementation.
  \item Production stage: many blockchains on the market are still young and premature. This problem is probably due to the fact that investors want to go fast on the market, creating overhead and confusion to the third party services that will embraced a yet-not-ready product.
  \item Contract execution: the execution of the contracts allows to define rules (e.g. which authorities is is able to change a GlobalId and the modalities of such operations).
  \item Decentralized Control: since it is a blockchain prerogative, the technology should not be owned by a restricted group of actors and everyone must be able to join at will.
\end{itemize}

In table \ref{table:1} we analyse the most famous blockchains.

\begin{table}[h!]
\centering
\begin{tabular}{ |p{7cm}|p{2cm}|p{2cm}|p{2cm}|p{2cm}|  }
\hline
\multicolumn{5}{|c|}{Chosing the blockchain} \\
\hline
Blockchain candidate & Security & Production Stage & Contract Execution & Public \\
\hline
Bitcoin and AltCoin \cite{Nakamoto_bitcoin:a} & Yes & Yes & No\cite{Bitcoin_notTuringComplete} & Yes \\
\hline
Ethereum \cite{ethereum_whitepaper} & Yes & Yes & Yes & Yes \\
\hline
BigchainDB \cite{bigchaindb_whitepaper} & No \cite{_bigchaindb_bullshit} \ & No & Yes & Yes \\
\hline
Lisk (not whitepaper available a.t.m., https://lisk.io/)& Yes & No \cite{lisk_problems} & Yes & Yes \\
\hline
IOTA \cite{iota_whitepaper}& No \cite{iota_problems} & No & Yes & Yes \\
\hline
Hyperledger Fabric \cite{martindale_fabric:_2017}& Yes & Yes & Yes & No \\
\hline
R3 Corda \cite{corda_whitepaper}& Yes & Yes & Yes & No \\
\hline
\end{tabular}
\caption{Table to compare blockchains}
\label{table:1}
\end{table}

Given this constrain, the Ethereum technology seems to be the one that most fits our necessities of replacing the previous DHT system.

\begin{notation}
  TODO: \\
  Explain why ethereum and make comaprison with bitcoin (simply: bitcoin is expensive and non optimazed for key value. Other blockchain are not safe => bigchaindb or not yet fully developed => Lisk).

  Other blockchain are not worth to be mention for lack of documentation.

  How to explain this without sources if not the persona experience??
\end{notation}
The main reasons we choose this approach are:

\begin{itemize}
  \item Consistent storage by definition. Ethereum blockchain creates different replicas of the same database in each node of the network. With such solution, nodes can enter or exit the network without compromising any social record.
  \item In case of failure, it is possible to recover the state of the database and re-execute the missing transactions.
  \item Ethereum is a distributed ledger, it is not possible to forge part of the database (at least 50\% + 1 of the nodes must be corrupted). Preventing any node to send misleading social records.
  \item One of the main features of the blockchain is the endless scalability: this permits to set up a high number of nodes (multiple GSLS servers per each OSN and the users who are willing to set up one node) without facing any delay.
  Even better, increasing the size of the network and therefore the difficulty of a rewrite attack (50\% + number of nodes).
  \item Hosting one node allows to locally access the social records stored into the GSLS storage. In the previous implementation (with DHT) it was possible that one specific social record was only available in nodes far away from the requesting resource, creating a delay in the delivery of the information. 
\end{itemize}

Given these premises, Ethereum blockchain seems to be the tool for the job: providing a consistent, distribute and scalable and environment; accessible by all the player at the same time.


\subsection{Ethereum ecosystem}

Ethereum is an \textit{account model} blockchain \cite{ethereum_whitepaper}, therefor different from the mainstream \textit{UTXO model}  (such as Bitcoin blockchain). This alternative implementation allows to save memory space since the balance is kept into the account state (a key-value storage mapping addresses to account state object \cite{ethereum_yellowpaper}) rather then keeping track of all the transaction changes (utxo) generated from one account.

Blockchain can be seen as a system that shift from a state $t$ ($\sigma_t$) to a state $t+1$ ($\sigma_{t+1}$) thanks to the transactions $[t_0,t_1,...,t_n]$ (where $n$ is the number of the transaction $n-1$) stored into the block $B$ \ref{table:2}.


\begin{table}[h!]
\centering
\begin{tabular}{ |p{3cm}|  }
\hline
\multicolumn{1}{|c|}{Block structure} \\
\hline
\hline
Header \\
\hline
Transactions $[t_0,t_1,...,t_n]$ \\
\hline
Headers of onners blocks\\
\hline
\end{tabular}
\caption{Basic structure of an Ethereum block}
\label{table:2}
\end{table}

Once the block is mined, is broadcasted and each node at state $t$  executes, through the Ethereum Virtual Machine (EVM), each transaction inside the block in order.
This algorithm allows the nodes to eventually converge to the same state $t+1$ at some point in time.

Building the transaction instead, is out of the scope of the EVM: this operations is delegated to the human being who owns the account or any sort of software/wallet she decides to employ.

There are two types of transactions: \textit{account creation} and \textit{message call to existing account}. Even though these two types are conceptually distant, they share the same structure:

\begin{itemize}
  \item \textbf{nonce}: number of transaction sent from the sender account.
  \item \textbf{gasPrice}: number of wei (smallest monetary unit) to pay per each unit of gas.
  \item \textbf{gasLimit}: maximum amount of gas that shall be used to execute the current transaction.
  \item \textbf{to}: address account of the receiver.
  \item \textbf{value}: number of wei assigned to the transaction.
  \item \textbf{v, r, s}: Value of the transaction signature, this parameters are used to recover the sender signature \cite{gura2004comparing}.
  \item \textbf{init}: field with unlimited size, here the initialisation commands are specified directly in EVM code. This field is analysed only if the \textit{to} field is left empty.
  \item \textbf{data}: field with unlimited size, specify the input data of the message call.
\end{itemize}

Is trivial to notice that among all these fields the \textit{v}, \textit{r} and \texti{s} are the most private ones; basing on this assumption we aim to create a workflow which allows the sender to create and upload transactions even when an internet connection is not available to him. 

\subsubsection{Serializability of the Ethereum transaction}
Each user has a public key and a private key. It is possible to infer the public key starting from the private key; so that, given a random sequence of bytes is possible to create the key pair.

The account address are just the last 160 bit of the public key hashed via SHA3-256 \cite{sha3_256_keccak}.

Once the values of the transaction are set: \textit{nonce}, \textit{gasPrice}, \textit{gasLimit}, \textit{to} (can be left empty if is a new contract), \textit{value}, \textit{data} is possible to serialise it via recursive length prefix (RLP) \cite{ethereum_whitepaper} and hash it. Then the hash is signed with the private key, generating \textit{v}, \textit{r}, \textit{s}.

\subsection{Verify correctness of the transaction}
Once these transaction is ready, it can be uploaded to the network and being mined.

When the EVM will execute it, it will in order:
\begin{enumerate}
  \item Verify that the RLP serialisation is not corrupted (no bytes at the end of the bytestream).
  \item Verify the transaction signature, recovering the public key from \textit{v}, \textit{r}, \textit{s} and finding the transaction hash; then comparing it with the RLP hash of the data received.
  \item Verify the validity od the \textit{nonce}, this operation can only be done after the public key has been recovered.
  \item Verify that the unit of gas are sufficient to complete the transaction.
  \item Verify that the sender balance is sufficient to execute the transaction ($gasPrice * gasLimit$).
\end{enumerate}

As showed, the transaction is serialised and encrypted when is uploaded to the blockchain.
At this stage is not possible to decrypt it or change any fields.

It is therefore possible to stream the bytes of the transaction from one device to another in a safe manner.

Put the picture made with illustrator which is self explanatory

\begin{notation}
  Need to maintain the account key pair in order to validate the social record
\end{notation}  % research questions and hypotheses
% \section{Methods}
% \label{S:4}


% The project will use the analytic method since it must follow the GSLS and the selected blockchain specifications in order to create a reliable and consistent transaction. The transaction should
% be created and sent without loss of data impact on the users’device as well as be safe and secure.

% \subsection{Analytic method}

% to do...       % methods use to validate our assumptionsap
\section{Evaluation}
\label{S:4}


\subsection{Security}

\subsubsection{In compare with the previous GSLS}

\begin{itemize}
  \item Now data are sent in an encrypted way - Communication channel is secure
  \item Data are 'more distributed' not only in the GSLS network but over all the nodes 
  \item Not possible to create replication/replay attack (or whatever is called, where you basically repeat the same instruction)
  \item Built in versioning system
  \item Nonce problem (one node could always return a wrong nonce), change the GSLS server
\end{itemize}

\subsubsection{In compare with light client}
\begin{itemize}
  \item This is a light client, just very light, it allows only few operations (not possible to hack it with other transactions if not the one we wrote in the abi file ) <-- confirm that
  \item All the security of a cold wallet (no 3rd party knows it), but still able to create transactions
  \item No need to install particular and specific blockchain nodes or to run any particular program in background. The Identity management over the Social Networks is a service widely used by the majority of the developed countries therefore must be easy to use and plug & play. [in our case you register as if a normal website -> with the same complexity for the user]
\end{itemize}   % result and analysis 
\section{Discussion}
\label{S:6}

\begin{answer}
    to do... \\
\end{answer}

\subsection{Pros and Cons}

\begin{answer}
    to do... \\
\end{answer}

\subsection{Performance compared to the current system}

\begin{answer}
    to do... \\
\end{answer}

\subsection{Future work}   % discussion


\bibliographystyle{IEEEtran}
%%\bibliographystyle{myIEEEtran}

\bibliography{bibliography/bib}
\appendix
% \section{Insensible Approximation}

% Note that the Appendix or Appendices are Optional.


\end{document}