%% This is file `elsarticle-template-1-num.tex',
%%
%% Copyright 2009 Elsevier Ltd
%%
%% This file is part of the 'Elsarticle Bundle'.
%% ---------------------------------------------
%%
%% It may be distributed under the conditions of the LaTeX Project Public
%% License, either version 1.2 of this license or (at your option) any
%% later version.  The latest version of this license is in
%%    http://www.latex-project.org/lppl.txt
%% and version 1.2 or later is part of all distributions of LaTeX
%% version 1999/12/01 or later.
%%
%% The list of all files belonging to the 'Elsarticle Bundle' is
%% given in the file `manifest.txt'.
%%
%% Template article for Elsevier's document class `elsarticle'
%% with numbered style bibliographic references
%%
%% $Id: elsarticle-template-1-num.tex 149 2009-10-08 05:01:15Z rishi $
%% $URL: http://lenova.river-valley.com/svn/elsbst/trunk/elsarticle-template-1-num.tex $
%%
\documentclass[preprint,12pt]{elsarticle}

%% Use the option review to obtain double line spacing
%% \documentclass[preprint,review,12pt]{elsarticle}

%% Use the options 1p,twocolumn; 3p; 3p,twocolumn; 5p; or 5p,twocolumn
%% for a journal layout:
%% \documentclass[final,1p,times]{elsarticle}
%% \documentclass[final,1p,times,twocolumn]{elsarticle}
%% \documentclass[final,3p,times]{elsarticle}
%% \documentclass[final,3p,times,twocolumn]{elsarticle}
%% \documentclass[final,5p,times]{elsarticle}
%% \documentclass[final,5p,times,twocolumn]{elsarticle}

%% if you use PostScript figures in your article
%% use the graphics package for simple commands
%% \usepackage{graphics}
%% or use the graphicx package for more complicated commands
%% \usepackage{graphicx}
%% or use the epsfig package if you prefer to use the old commands
%% \usepackage{epsfig}

%% The amssymb package provides various useful mathematical symbols
\usepackage{amssymb}
%% The amsthm package provides extended theorem environments
%% \usepackage{amsthm}

%% The lineno packages adds line numbers. Start line numbering with
%% \begin{linenumbers}, end it with \end{linenumbers}. Or switch it on
%% for the whole article with \linenumbers after \end{frontmatter}.
\usepackage{lineno}

\usepackage[utf8]{inputenc}
\usepackage[T1]{fontenc} % Use 8-bit encoding that has 256 glyphs
\usepackage{fourier} % Use the Adobe Utopia font for the document - comment this line to return to the LaTeX default
\usepackage[english]{babel} % English language/hyphenation
\usepackage{amsmath,amsfonts,amsthm} % Math packages


\usepackage{rotating}
\newcommand\tabrotate[1]{\begin{turn}{90}\rlap{#1}\end{turn}}


 % Useful Basics
 \usepackage{color}
 \usepackage{xcolor}
 \usepackage{framed}
 \usepackage{listings}

 % Write Answer
 \newenvironment{notation}{%
   \def\FrameCommand{\colorbox{yellow!20}}%
   \MakeFramed{\advance\hsize-\width \FrameRestore}}
 {\endMakeFramed}

%% natbib.sty is loaded by default. However, natbib options can be
%% provided with \biboptions{...} command. Following options are
%% valid:

%%   round  -  round parentheses are used (default)
%%   square -  square brackets are used   [option]
%%   curly  -  curly braces are used      {option}
%%   angle  -  angle brackets are used    <option>
%%   semicolon  -  multiple citations separated by semi-colon
%%   colon  - same as semicolon, an earlier confusion
%%   comma  -  separated by comma
%%   numbers-  selects numerical citations
%%   super  -  numerical citations as superscripts
%%   sort   -  sorts multiple citations according to order in ref. list
%%   sort&compress   -  like sort, but also compresses numerical citations
%%   compress - compresses without sorting
%%
%% \biboptions{comma,round}

% \biboptions{}

\setlength\parindent{0pt} % Removes all indentation from paragraphs - comment this line for an assignment with lots of text

\usepackage[backend=bibtex,natbib=true]{biblatex}
%\usepackage[backend=bibtex,style=ieee,natbib=true]{biblatex}
%\usepackage[backend=bibtex,style=authoryear,natbib=true]{biblatex} % Use the bibtex backend with the authoryear citation style (which resembles APA)
\bibliography{bibliography/bib.bib}
%\addbibresource{bibliography/bib}


\journal{}

\begin{document}

\begin{frontmatter}

%% Title, authors and addresses

%% use the tnoteref command within \title for footnotes;
%% use the tnotetext command for the associated footnote;
%% use the fnref command within \author or \address for footnotes;
%% use the fntext command for the associated footnote;
%% use the corref command within \author for corresponding author footnotes;
%% use the cortext command for the associated footnote;
%% use the ead command for the email address,
%% and the form \ead[url] for the home page:
%%
%% \title{Title\tnoteref{label1}}
%% \tnotetext[label1]{}
%% \author{Name\corref{cor1}\fnref{label2}}
%% \ead{email address}
%% \ead[url]{home page}
%% \fntext[label2]{}
%% \cortext[cor1]{}
%% \address{Address\fnref{label3}}
%% \fntext[label3]{}

\title{Identity Management in Blockchain}


%% use optional labels to link authors explicitly to addresses:
%% \author[label1,label2]{<author name>}
%% \address[label1]{<address>}
%% \address[label2]{<address>}

\author{Riccardo Sibani, Filippo Boiani}

\address{KTH, Kungliga Tekniska Högskolans} 


\begin{abstract}
%% Text of abstract
Suspendisse potenti. Suspendisse quis sem elit, et mattis nisl. Phasellus consequat erat eu velit rhoncus non pharetra neque auctor. Phasellus eu lacus quam. Ut ipsum dolor, euismod aliquam congue sed, lobortis et orci. Mauris eget velit id arcu ultricies auctor in eget dolor. Pellentesque suscipit adipiscing sem, imperdiet laoreet dolor elementum ut. Mauris condimentum est sed velit lacinia placerat. Vestibulum ante ipsum primis in faucibus orci luctus et ultrices posuere cubilia Curae; Nullam diam metus, pharetra vitae euismod sed, placerat ultrices eros. Aliquam tincidunt dapibus venenatis. In interdum tellus nec justo accumsan aliquam. Nulla sit amet massa augue.
\end{abstract}

\begin{keyword}
Science \sep Publication \sep Complicated
%% keywords here, in the form: keyword \sep keyword

%% MSC codes here, in the form: \MSC code \sep code
%% or \MSC[2008] code \sep code (2000 is the default)

\end{keyword}

\end{frontmatter}     % abstract and keywords
\section{Introduction}
\label{S:1}
 % introduction to the topic 
\section{Theoretical Framework and Literature Study}
\label{S:2}

Some text here.

\subsection{The Sonic project}

The way people enjoy the news and interact within their network and friends changed with the advent of the OSN (Online Social network) platforms. \par
Nowadays, OSNs are the main communication media but the information they generate are proprietary and, because of it, it is easy to imagine the reason behind why they try to create the so called \textit{lock in} effect.
In fact, the information about the users generated by surfing on the platform, once collected, allows the OSN to create personalized advertisements. Such behaviour, incentivises the social network platforms to implement barriers to exit. \par
What SONIC aims to reach is to connect all the platforms in a ``decentralized and ethereogeneus federation of OSN platforms'' (http://sonic-project.net/), via a specific protocol which allows to migrate and interact with people registered among different OSNs \cite{gondor_sonic:_2014}. \par
User identification is performed via a generated Unified Unique Identifier (UUID) derived from a \textit{PKCS\#8} \cite{pkcs8}with 8 random bytes added at the end, which is then decrypted with \textit{PBKDF\#2} \cite{pkcs8} with settings SHA256 \cite{hansen_us} for 10 000 iterations.
The output is finally a 256bit long field which is converted in base36, generating the definitive UUID. \par

Example of a UUID is: \\ \textbf{3R2IWN230NFI2QBYUDEQW02134DBSUIBPPOFWCDIN221343EEE} \par

Accounts are stored in a distributed directory service named \textit{Global Social Lookup System} which use a DHT in order to store and retrieve the single accounts. In particular, the accounts are hashed and based on the hash stored in a subset of GSLS Servers (consistent hashing). \par

Each user has 2 asymmetric keys: \textit{PersonalKeyPair} and \textit{AccountKeyPair}. The former to sign the registration and update the account information in the DTH, the latter signs and verify every payload exchange through the SONIC implementation.

\subsection{Blockchain}

DHT as a storage layer seemed to be the best solution in the previous implementation, but unfortunately the adopted framework TOMP2P \cite{tomp2p:2017} does not provide any functionality to validate the peer transmissions. 
This lack results in the possibility to send malicious information (such as delete users).
Thus, given the open source distribution licence, the risk of such attack is higher since anyone can set up its own GSLS node and join the network: compromising the entire system. \par

Another secondary problem is the distributed nature of the DHT which, in the current system, distributes the records with indirect replication \cite{_tomp2p_2017} (a certain number \textit{n} of nodes store the same social record). Nevertheless the users' information must be quickly retrievable by the main OSNs and in the current solution there may be some delays. \par

One last issue of the current GSLS is that the records are store in the memory (RAM) and in case of failure or switch off of the GSLS server, all the records on that server will be lost. \par

\subsection{Requirements of the solution}
The features that the new implementation requires:
\begin{itemize}
  \item Consistent and distributed storage, able to eventually converge to the same state and to persist in case of failures.
  \item A secure procedure to create, update and delete the social records of the users. The new solution must, in addition, be able to guarantee the security of the database even in case of a malicious attack or faulty behavior of the peer nodes.
  \item The new solution should be able to guarantee a faster lookup of the records based on the UUID key. This will allow to quickly provide the user's information to the requesting OSN or user. It may seem a secondary issue but, in a live environment, in order to convince the major OSNs to join such system, the solution should not reflect negatively on the performance.
\end{itemize}

To summarize, the project aims to find a solution able to guarantee security and agreement in an open source environment, where different competitors share the same data structure (meaning the social record). \par











% Short problem, 
% Hypothesis,
% a bit of what we will do and what we expect as outcome,

% The blockchain technology, born in 2009 from \textbf{Satoshi Nakamoto} [check if it is spelled correctly and place a reference to the paper], is still in its embryonal phase and its possibilities are yet not entirely explored.

% The main feature of this distributed ledger is to eventually reach an unanimous agreement among the different ledgers guaranteeing the integrity of the data. 

% Scalability is another key feature of the blockchain, each and every ledger act like a node in a distributed system: meaning that an application running on a server can easily access the local node with almost no latency. \textbf{<- Latency? }
% \\
% Given these premises, blockchain seems to be the right tool for the job: providing a consistent, distribute and scalable and environment, accessible by all the player at the same time.






% \subsubsection{Sonic}
% \subsubsection{GSLS}


% \subsection{Directory service}



% \subsection{DHT with P2P networks}

% \subsubsection{DHT}

% \subsubsection{Encription over the communication channels}




% \subsection{Blockchain}


% \subsubsection{Transaction centric}
% \subsubsection{State centric}



% \subsection{Blockchains on the market}

% \subsubsection{Bitcoin}

% \subsubsection{Ethereum}

% \subsubsection{Hyperledger - Fabric}

% \subsubsection{BigchainDb}
% [No bft]



% \subsection{Fruition of the blockchain}

% \subsubsection{Hot Wallet}
% \subsubsection{Cold Wallet}

% \subsubsection{Light clients}

% \subsection{Blockstack as an example of blockchain utilization in Open Source enviroment}
% Blockstack is a particular implementation of a decentralized DNS system based on blockchain. It combines DNS functionality with public key infrastructure and is primarily meant to be used by new blockchain applications.

% According to the company: \enquote{under the hood, Blockstack provides a decentralized domain name system (DNS), decentralized public key distribution system, and registry for apps and user identities} \cite{BlockStackMainPage}.

% The real breakthrough is the architecture the system is built on. It can be described as a three-layer design with the blockchain as the first and lower tier, the storage system as the upper and the peer network as middle layer.

% \subsection{Encryption}
% \subsubsection{Elliptic Curve}
% secp256k1
% https://crypto.stackexchange.com/questions/18965/is-secp256r1-more-secure-than-secp256k1

% \subsubsection{Asymmetric Encription <-- Needed????}



 % theoretical framework and literature study 
\section{The case for GSLS System}
\label{S:3}

\section{Security issues}
\begin{itemize}
  \item No verify the identity of the uploader in the DHT (one node can say someone updated, pass the payload and the other nodes will just register the change). Unfortunately change this behaviour is not very easy due to the current implementation of the DHT, based on a Peer2Peer framework which does not allow to perform such operations in a peer network (you have to trust the sender peer).
  \item What we want to do with the blockchain is to insert an additional level of security.
\end{itemize}



\subsection{Chose the most appropriate blockchain}



\subsection{How the users will use the Identity Management System}
non invasive approach\\
no light client etc -> better, we develop a super light client able to only perform 2 operations (read and modify/create record)


\subsection{Structure of the account}

\subsubsection{Structure of the Ethereum Transaction}

raw transaction\\

describe all the components

\subsubsection{Serializability of the Ethereum Transaction}

Put the picture made with illustrator which is self explanatory  % research questions and hypotheses
% \section{Methods}
% \label{S:4}


% The project will use the analytic method since it must follow the GSLS and the selected blockchain specifications in order to create a reliable and consistent transaction. The transaction should
% be created and sent without loss of data impact on the users’device as well as be safe and secure.

% \subsection{Analytic method}

% to do...       % methods use to validate our assumptionsap
\section{Evaluation}
\label{S:4}


\subsection{Security}

\subsubsection{In compare with the previous GSLS}

\begin{itemize}
  \item Now data are sent in an encrypted way - Communication channel is secure
  \item Data are 'more distributed' not only in the GSLS network but over all the nodes 
  \item Not possible to create replication/replay attack (or whatever is called, where you basically repeat the same instruction)
  \item Built in versioning system
  \item Nonce problem (one node could always return a wrong nonce), change the GSLS server
\end{itemize}

\subsubsection{In compare with light client}
\begin{itemize}
  \item This is a light client, just very light, it allows only few operations (not possible to hack it with other transactions if not the one we wrote in the abi file ) <-- confirm that
  \item All the security of a cold wallet (no 3rd party knows it), but still able to create transactions
  \item No need to install particular and specific blockchain nodes or to run any particular program in background. The Identity management over the Social Networks is a service widely used by the majority of the developed countries therefore must be easy to use and plug & play. [in our case you register as if a normal website -> with the same complexity for the user]
\end{itemize}   % result and analysis 
\section{Discussion}
\label{S:6}

\begin{notation}
    to do... \\
\end{notation}

\subsection{Pros and Cons}

\begin{notation}
    to do... \\
\end{notation}

\subsection{Performance compared to the current system}

\begin{notation}
    to do... \\
\end{notation}

\subsection{Future work}   % discussion



%%
%% Start line numbering here if you want
%%
% \linenumbers

%% main text
% \section{The First Section}
% \label{S:1}

% Maecenas \cite{Smith:2012qr} fermentum \cite{Smith:2013jd} urna ac 

% \begin{itemize}
% \item Bullet point one
% \item Bullet point two
% \end{itemize}

% \begin{enumerate}
% \item Numbered list item one
% \item Numbered list item two
% \end{enumerate}

% \subsection{Subsection One}

% Quisque elit ipsum, porttitor et imperdiet in, facilisis ac diam. Nunc 

% \begin{table}[h]
% \centering
% \begin{tabular}{l l l}
% \hline
% \textbf{Treatments} & \textbf{Response 1} & \textbf{Response 2}\\
% \hline
% Treatment 1 & 0.0003262 & 0.562 \\
% Treatment 2 & 0.0015681 & 0.910 \\
% Treatment 3 & 0.0009271 & 0.296 \\
% \hline
% \end{tabular}
% \caption{Table caption}
% \end{table}

% \subsection{Subsection Two}

% Donec eget ligula venenatis est posuere eleifend in sit amet diam. 

% \begin{figure}[h]
% \centering\includegraphics[width=0.4\linewidth]{placeholder}
% \caption{Figure caption}
% \end{figure}

% Integer risus dui, condimentum et gravida vitae, adipiscing et enim. 

% \begin{equation}
% \label{eq:emc}
% e = mc^2
% \end{equation}

% \section{The Second Section}
% \label{S:2}

% Reference to Section \ref{S:1}. Etiam congue sollicitudin diam non 

%% The Appendices part is started with the command \appendix;
%% appendix sections are then done as normal sections
%% \appendix

%% \section{}
%% \label{}

%% References
%%
%% Following citation commands can be used in the body text:
%% Usage of \cite is as follows:
%%   \cite{key}          ==>>  [#]
%%   \cite[chap. 2]{key} ==>>  [#, chap. 2]
%%   \citet{key}         ==>>  Author [#]

%% References with bibTeX database:

% \bibliographystyle{model1-num-names}
% \bibliography{bib.bib}

%% Authors are advised to submit their bibtex database files. They are
%% requested to list a bibtex style file in the manuscript if they do
%% not want to use model1-num-names.bst.

%% References without bibTeX database:

% \begin{thebibliography}{00}

% % \bibitem must have the following form:
% %   \bibitem{key}...
% %

% \bibitem{}

% \end{thebibliography}
\printbibliography[heading=bibintoc]


\end{document}

%%
%% End of file `elsarticle-template-1-num.tex'.
