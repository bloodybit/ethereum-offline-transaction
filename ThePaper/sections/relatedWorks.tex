\section{Theoretical Framework and Literature Study}
\label{S:2}

\subsection{The Sonic project}

The way people enjoy the news and interact within their network and friends changed with the advent of the OSN (Online Social network) platforms. \par
Nowadays, OSNs are the main communication media but the information they generate are proprietary and, because of it, it is easy to imagine the reason behind why they try to create the so called \textit{lock in} effect.
In fact, the information about the users generated by surfing on the platform, once collected, allows the OSN to create personalized advertisements. Such behaviour, incentivises the social network platforms to implement barriers to exit. \par
What SONIC \cite{gondor_sonic:_2014} aims to reach is to connect all the platforms in a ``decentralized and ethereogeneus federation of OSN platforms'', via a specific protocol which allows to migrate and interact with people registered among different OSNs. \par
User identification is performed via a generated Unified Unique Identifier (UUID) derived from a \textit{PKCS#8} with 8 random bytes added at the end, which is then decrypted with \textit{PBKDF#2} with settings SHA256 for 10 000 iterations.
The output is finally a 256bit long field which is converted in base36, generating the definitive UUID. \par

Example of a UUID is: \\ \textbf{3R2IWN230NFI2QBYUDEQW02134DBSUIBPPOFWCDIN221343EEE} \par

The accounts are stored in a distributed directory service named \textit{Global Social Lookup System} (GSLS) which use a Distributed Hash Table (DHT) in order to store and retrieve the single accounts. In particular, the accounts are hashed and based on the hash stored in a subset of GSLS Servers (consistent hashing). \par

Each user has 2 asymmetric keys: \textit{PersonalKeyPair} and \textit{AccountKeyPair}. The former is in charge to sign the registration and update of the account information in the DTH, the latter instead signs and verify every payload exchange through the SONIC implementation.

\subsection{Blockchain}

The DHT as a storage layer seemed to be the best solution back at the time, but unfortunately the implementation framework (TOMP2P) does not provide any functionality for the validation of the peer transmissions. 
This lack results in the possibility to send malicious information (such as delete users).
Thus, given the open source distribution licence, the risk of such attack is even higher since anyone can set up its own GSLS node and join the network: compromising the entire system. \par

Another secondary problem is the distributed nature of the DHT which, in the current system, distributes the records with indirect replication \cite{_tomp2p_2017} (a certain number \textit{n} of nodes store the same social record). Nevertheless the users' information must be quickly retrievable by the main OSNs and in the current solution there may be some slow downs. \par

One last issue of the current GSLS is that the records are store in the memory (RAM) and in case of failure or the turning off of the GSLS server, all the records on that server will be lost. \par

\subsection{Requirements of the solution}
The features that the new implementation requires are the following:
\begin{itemize}
  \item Consistent and distributed storage, able to eventually converge on the same state and to persist it in case of failures.
  \item A secure procedure to create, update and delete the social records of the users. The new solution must, in addition, be able to guarantee the security of the database even in case of a malicious attack or faulty behavior of the peer nodes.
  \item Possibly, the new solution should be able to guarantee a faster lookup of the records based on the UUID key. This will allow to quickly provide the information of each user to the requesting OSN or user. It may seem a secondary issue but, in a live environment, in order to convince the major OSNs to join such system, the solution should not reflect negatively on the performance.
\end{itemize}

To summarize, the project aims to find a solution able to guarantee security and agreement in an open source environment, where different competitors share the same data structure (meaning the social record). \par











% Short problem, 
% Hypothesis,
% a bit of what we will do and what we expect as outcome,

% The blockchain technology, born in 2009 from \textbf{Satoshi Nakamoto} [check if it is spelled correctly and place a reference to the paper], is still in its embryonal phase and its possibilities are yet not entirely explored.

% The main feature of this distributed ledger is to eventually reach an unanimous agreement among the different ledgers guaranteeing the integrity of the data. 

% Scalability is another key feature of the blockchain, each and every ledger act like a node in a distributed system: meaning that an application running on a server can easily access the local node with almost no latency. \textbf{<- Latency? }
% \\
% Given these premises, blockchain seems to be the right tool for the job: providing a consistent, distribute and scalable and environment, accessible by all the player at the same time.






% \subsubsection{Sonic}
% \subsubsection{GSLS}


% \subsection{Directory service}



% \subsection{DHT with P2P networks}

% \subsubsection{DHT}

% \subsubsection{Encription over the communication channels}




% \subsection{Blockchain}


% \subsubsection{Transaction centric}
% \subsubsection{State centric}



% \subsection{Blockchains on the market}

% \subsubsection{Bitcoin}

% \subsubsection{Ethereum}

% \subsubsection{Hyperledger - Fabric}

% \subsubsection{BigchainDb}
% [No bft]



% \subsection{Fruition of the blockchain}

% \subsubsection{Hot Wallet}
% \subsubsection{Cold Wallet}

% \subsubsection{Light clients}

% \subsection{Blockstack as an example of blockchain utilization in Open Source enviroment}
% Blockstack is a particular implementation of a decentralized DNS system based on blockchain. It combines DNS functionality with public key infrastructure and is primarily meant to be used by new blockchain applications.

% According to the company: \enquote{under the hood, Blockstack provides a decentralized domain name system (DNS), decentralized public key distribution system, and registry for apps and user identities} \cite{BlockStackMainPage}.

% The real breakthrough is the architecture the system is built on. It can be described as a three-layer design with the blockchain as the first and lower tier, the storage system as the upper and the peer network as middle layer.

% \subsection{Encryption}
% \subsubsection{Elliptic Curve}
% secp256k1
% https://crypto.stackexchange.com/questions/18965/is-secp256r1-more-secure-than-secp256k1

% \subsubsection{Asymmetric Encription <-- Needed????}



