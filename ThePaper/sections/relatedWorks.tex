\section{Background}
\label{S:2}

put related work inside Introduction?? [and call it Background]

\subsection{The Sonic project}

The way people enjoy the news and interact within their network and friends changed with the advent of the OSN (Online Social network) platforms. \par
Nowadays, OSNs are the main communication media but the information they generate are proprietary and, because of it, it is easy to imagine the reason behind why they try to create the so called \textit{lock in} effect.
In fact, the information about the users generated by surfing on the platform, once collected, allows the OSN to create personalized advertisements. Such behaviour, incentivise the social network platforms to implement barriers to exit. \par
What SONIC \cite{gondor_sonic:_2014} aims to reach is to connect all the platforms in a ``decentralized and ethereogeneus federation of OSN platforms'', connect via a specific protocol which allows to migrate and interact with people registered among different OSNs.

\subsubsection{Sonic}
\subsubsection{GSLS}


\subsection{Directory service}



\subsection{DHT with P2P networks}

\subsubsection{DHT}

\subsubsection{Encription over the communication channels}




\subsection{Blockchain}


\subsubsection{Transaction centric}
\subsubsection{State centric}



\subsection{Blockchains on the market}

\subsubsection{Bitcoin}

\subsubsection{Ethereum}

\subsubsection{Hyperledger - Fabric}

\subsubsection{BigchainDb}
[No bft]



\subsection{Fruition of the blockchain}

\subsubsection{Hot Wallet}
\subsubsection{Cold Wallet}

\subsubsection{Light clients}

\subsection{Blockstack as an example of blockchain utilization in Open Source enviroment}
Blockstack is a particular implementation of a decentralized DNS system based on blockchain. It combines DNS functionality with public key infrastructure and is primarily meant to be used by new blockchain applications.

According to the company: \enquote{under the hood, Blockstack provides a decentralized domain name system (DNS), decentralized public key distribution system, and registry for apps and user identities} \cite{BlockStackMainPage}.

The real breakthrough is the architecture the system is built on. It can be described as a three-layer design with the blockchain as the first and lower tier, the storage system as the upper and the peer network as middle layer.

\subsection{Encryption}
\subsubsection{Elliptic Curve}
secp256k1
https://crypto.stackexchange.com/questions/18965/is-secp256r1-more-secure-than-secp256k1

\subsubsection{Asymmetric Encription <-- Needed????}



