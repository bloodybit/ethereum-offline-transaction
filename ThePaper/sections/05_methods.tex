\section{Methods}
\label{S:4}

We investigated different blockchain implementations and compared their features to gain an insight of what could be the most suitable solution. We eventually decided upon Ethereum as it offers a turing-complete language for smart contracts along with a great community of developers and a extensive documentation. Of course, this choice has also some side effectes, in this section we describe our steps towards the decision and the implementation of the system.  

\subsection{Choosing the blockchain}

The choice of the most appropriate blockchain technology should take into account:
\begin{itemize}
  \item Acknowledged security: the blockchain should not contain any technical flaw and be recognized by the community as \textit{trustable}\footnote{The authors realise that this concept is abstract and often biased, in order to evaluate the \textit{trustability}, github stars (when available) and blockchain market capitalization value (when available) have been analysed} . This is important since many projects are based on a good and promising concept but often times fail on the practical implementation.
  \item Production stage: many blockchains on the market are still young and premature. This problem is probably due to the fact that investors want to go fast on the market, creating overhead and confusion to the third party services that will embraced a yet-not-ready product.
  \item Contract execution: the execution of the contracts allows to define rules (e.g. which authorities is able to change a GlobalId and the modalities of such operations).
  \item Decentralized Control: since it is a blockchain prerogative, the technology should not be owned by a restricted group of actors and everyone must be able to join at will.
\end{itemize}

In table \ref{table:1} we analyse the most famous blockchains.

\begin{table}[h!]
\centering
\begin{tabular}{ |p{7cm}|p{2cm}|p{2cm}|p{2cm}|p{2cm}|  }
\hline
\multicolumn{5}{|c|}{Chosing the blockchain} \\
\hline
Blockchain candidate & Security & Production Stage & Contract Execution & Public \\
\hline
Bitcoin and AltCoins \cite{Nakamoto_bitcoin:a} & Yes & Yes & No\cite{Bitcoin_notTuringComplete} & Yes \\
\hline
Ethereum \cite{ethereum_whitepaper} & Yes & Yes & Yes & Yes \\
\hline
BigchainDB \cite{bigchaindb_whitepaper} & No \cite{_bigchaindb_bullshit} \ & No & Yes & Yes \\
\hline
Lisk (not whitepaper available a.t.m., https://lisk.io/)& Yes & No \cite{lisk_problems} & Yes & Yes \\
\hline
IOTA \cite{iota_whitepaper}& No \cite{iota_problems} & No & Yes & Yes \\
\hline
Hyperledger Fabric \cite{martindale_fabric:_2017}& Yes & Yes & Yes & No \\
\hline
R3 Corda \cite{corda_whitepaper}& Yes & Yes & Yes & No \\
\hline
\end{tabular}
\caption{Table to compare blockchains}
\label{table:1}
\end{table}



Given this constrain, the Ethereum technology seems to be the one that most fits our necessities of replacing the previous DHT system.

% \begin{notation}
%   TODO: \\
%   Explain why ethereum and make comaprison with bitcoin (simply: bitcoin is expensive and non optimazed for key value. Other blockchain are not safe => bigchaindb or not yet fully developed => Lisk).

%   Other blockchain are not worth to be mention for lack of documentation.

%   How to explain this without sources if not the persona experience??
% \end{notation}
The main reasons we choose this approach are:

\begin{itemize}
  \item Consistent storage by definition. Ethereum blockchain creates different replicas of the same database in each node of the network. With such solution, nodes can enter or exit the network without compromising any social record.
  \item In case of failure, it is possible to recover the state of the database and re-execute the missing transactions.
  \item Ethereum is a distributed ledger, it is not possible to forge part of the database (at least 50\% + 1 of the nodes must be corrupted). Preventing any node to send misleading social records.
  \item One of the main features of the blockchain is the endless scalability: this allows to set up a high number of nodes (multiple GSLS servers for each OSN and for the users who are willing to set up one node) without facing any delay.
  Even better, increasing the size of the network and therefore the difficulty of a rewrite attack (50\% + number of nodes).
  \item Hosting one node allows to locally access the social records stored into the GSLS storage. In the previous implementation (with DHT) it was possible that one specific social record was only available in nodes far away from the requesting resource, creating a delay in the delivery of the information. 
\end{itemize}

Given these premises, Ethereum blockchain seems to be the tool for the job: providing a consistent, distribute and scalable and environment; accessible by all the player at the same time.


\subsection{Ethereum ecosystem}

Ethereum is an \textit{account model} blockchain \cite{ethereum_whitepaper}, therefore different from the mainstream \textit{UTXO model}  (such as Bitcoin blockchain). This alternative implementation allows to save memory space since the balance is kept into the account state (a key-value storage mapping addresses to account state object \cite{ethereum_yellowpaper}) rather than keeping track of all the transaction changes (utxo) generated from one account.

Blockchain can be seen as a system that shift from a state $t$ ($\sigma_t$) to a state $t+1$ ($\sigma_{t+1}$) thanks to the transactions $[t_0,t_1,...,t_n]$ (where $n$ is the number of the transaction $n-1$) stored into the block $B$ \ref{table:2}.


\begin{table}[h!]
\centering
\begin{tabular}{ |p{3cm}|  }
\hline
\multicolumn{1}{|c|}{Block structure} \\
\hline
\hline
Header \\
\hline
Transactions $[t_0,t_1,...,t_n]$ \\
\hline
Headers of onners blocks\\
\hline
\end{tabular}
\caption{Basic structure of an Ethereum block}
\label{table:2}
\end{table}

Once the block is mined, is broadcasted and each node at state $t$  executes, through the Ethereum Virtual Machine (EVM), each transaction inside the block in order.
This algorithm allows the nodes to eventually converge to the same state $t+1$ at some point in time.

Building the transaction instead, is out of the scope of the EVM: this operations is delegated to the human being who owns the account or any sort of software/wallet she decides to employ.

There are two types of transactions: \textit{account creation} and \textit{message call to existing account}. Even though these two types are conceptually distant, they share the same structure:

\begin{itemize}
  \item \textbf{nonce}: number of transactions sent from the sender account.
  \item \textbf{gasPrice}: number of wei (smallest monetary unit) to pay per each unit of gas.
  \item \textbf{gasLimit}: maximum amount of gas that shall be used to execute the current transaction.
  \item \textbf{to}: address account of the receiver.
  \item \textbf{value}: number of wei assigned to the transaction.
  \item \textbf{v, r, s}: Value of the transaction signature, this parameters are used to recover the sender signature \cite{gura2004comparing}.
  \item \textbf{init}: field with unlimited size, here the initialisation commands are specified directly in EVM code. This field is analysed only if the \textit{to} field is left empty.
  \item \textbf{data}: field with unlimited size, specify the input data of the message call.
\end{itemize}

Is trivial to notice that among all these fields the \textit{v}, \textit{r} and \textit{s} are the most private ones; basing on this assumption we aim to create a workflow which allows the sender to create and upload transactions even when an internet connection is not available to him. 

\subsubsection{Serializability of the Ethereum transaction}
Each user has a public key and a private key. It is possible to infer the public key starting from the private key; so that, given a random sequence of bytes is possible to create the key pair.

The account address are just the last 160 bit of the public key hashed via SHA3-256 \cite{sha3_256_keccak}.

Once the values of the transaction are set: \textit{nonce}, \textit{gasPrice}, \textit{gasLimit}, \textit{to} (can be left empty if is a new contract), \textit{value}, \textit{data} is possible to serialise it via recursive length prefix (RLP) \cite{ethereum_whitepaper} and hash it. Then the hash is signed with the private key, generating \textit{v}, \textit{r}, \textit{s}.

\subsection{Verify correctness of the transaction}
Once the transaction is ready, it can be uploaded to the network and being mined.

When the EVM will execute it, it will in order:
\begin{enumerate}
  \item Verify that the RLP serialisation is not corrupted (no bytes at the end of the bytestream).
  \item Verify the transaction signature, recovering the public key from \textit{v}, \textit{r}, \textit{s} and finding the transaction hash; then comparing it with the RLP hash of the data received.
  \item Verify the validity often the \textit{nonce}, this operation can only be done after the public key has been recovered.
  \item Verify that the unit of gas are sufficient to complete the transaction.
  \item Verify that the sender balance is sufficient to execute the transaction ($gasPrice * gasLimit$).
\end{enumerate}

As showed, the transaction is serialised and encrypted when is uploaded to the blockchain.
At this stage is not possible to decrypt it or change any fields.

It is therefore possible to stream the bytes of the transaction from one device to another in a safe manner.


\begin{notation}
  Put the picture made with illustrator which is self explanatory
  Need to maintain the account key pair in order to validate the social record
\end{notation}