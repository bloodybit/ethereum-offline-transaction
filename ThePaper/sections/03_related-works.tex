\section{Theoretical Framework and Literature Study}
\label{S:2}

In this section we take a closer look at how the SONIC project works and how blockchain could impove it. 

\subsection{The Sonic project}

The way people enjoy news and interact within their network of friends changed with the advent of the OSN platforms. \par
Today, OSNs are the main communication media, but the information they generate is proprietary and, the reason behind this is the desire to create the so called \textit{lock in} effect.
Moreover, the information about the users generated by surfing on the platform, once collected, allows the OSN to create personalized advertisements. Such behaviour incentivises the social network platforms to implement barriers to exit. \par
SONIC aims to connect all the platforms in a ``decentralized and hetherogeneous federation of OSN platforms'' \textbf{[http://sonic-project.net/] put as a reference}, via a specific protocol which allows to migrate and interact with people registered in different OSNs \cite{gondor_sonic:_2014}. \par
User identification is performed via a Unified Unique Identifier (UUID) derived from a \textit{PKCS\#8} \cite{pkcs8}with 8 random bytes added at the end, which is then decrypted with \textit{PBKDF\#2} \cite{pkcs8} with SHA256 \cite{hansen_us} for 10 000 iterations.
The output is a 256bit long value which is converted in base36, generating a definitive UUID. \par

An example of a UUID is: \\ \textbf{3R2IWN230NFI2QBYUDEQW02134DBSUIBPPOFWCDIN221343EEE}\cite{gondor_sonic:_2014} \par

\begin{notation}
Format the UUID properly
\end{notation}

Accounts are stored in a distributed directory service named \textit{Global Social Lookup System} using a DHT to store and retrieve the individual accounts. In particular, the accounts are hashed and based on the hash stored in a subset of GSLS Servers (consistent hashing). \par

Each user has 2 asymmetric keys: \textit{PersonalKeyPair} and \textit{AccountKeyPair}. The former is used to sign the registration and update the account information in the DTH, while the latter is used to sign and verify every payload exchanged through the SONIC implementation.

\subsection{Blockchain}

Using a DHT as a storage layer seemed to be the best solution in the previous implementation, but unfortunately TOMP2P framework \cite{tomp2p:2017} does not provide any functionality to validate the peer transmissions. 
This results in the possibility to send malicious information (such as delete users).
Given the open source distribution licence, the risk of such attack is high since anyone can set up their own GSLS node and join the network, thus compromising the entire system. \par

Another secondary problem is the distributed nature of the DHT, distributes the records with indirect replication \cite{_tomp2p_2017} (a certain number \textit{n} of nodes store the same SR). Ideally the users' information must be quickly retrievable by the main OSNs, but in the current solution there may be some delays when the SR is stored in a different GSLS. \par

One last issue of the current GSLS is that the SRs are stored in memory (RAM) and in case of failure or the GSLS server being switched off, all the records on that server will be lost. \par










% Short problem, 
% Hypothesis,
% a bit of what we will do and what we expect as outcome,

% The blockchain technology, born in 2009 from \textbf{Satoshi Nakamoto} [check if it is spelled correctly and place a reference to the paper], is still in its embryonal phase and its possibilities are yet not entirely explored.

% The main feature of this distributed ledger is to eventually reach an unanimous agreement among the different ledgers guaranteeing the integrity of the data. 

% Scalability is another key feature of the blockchain, each and every ledger act like a node in a distributed system: meaning that an application running on a server can easily access the local node with almost no latency. \textbf{<- Latency? }
% \\
% Given these premises, blockchain seems to be the right tool for the job: providing a consistent, distribute and scalable and environment, accessible by all the player at the same time.






% \subsubsection{Sonic}
% \subsubsection{GSLS}


% \subsection{Directory service}



% \subsection{DHT with P2P networks}

% \subsubsection{DHT}

% \subsubsection{Encription over the communication channels}




% \subsection{Blockchain}


% \subsubsection{Transaction centric}
% \subsubsection{State centric}



% \subsection{Blockchains on the market}

% \subsubsection{Bitcoin}

% \subsubsection{Ethereum}

% \subsubsection{Hyperledger - Fabric}

% \subsubsection{BigchainDb}
% [No bft]



% \subsection{Fruition of the blockchain}

% \subsubsection{Hot Wallet}
% \subsubsection{Cold Wallet}

% \subsubsection{Light clients}

% \subsection{Blockstack as an example of blockchain utilization in Open Source enviroment}
% Blockstack is a particular implementation of a decentralized DNS system based on blockchain. It combines DNS functionality with public key infrastructure and is primarily meant to be used by new blockchain applications.

% According to the company: \enquote{under the hood, Blockstack provides a decentralized domain name system (DNS), decentralized public key distribution system, and registry for apps and user identities} \cite{BlockStackMainPage}.

% The real breakthrough is the architecture the system is built on. It can be described as a three-layer design with the blockchain as the first and lower tier, the storage system as the upper and the peer network as middle layer.

% \subsection{Encryption}
% \subsubsection{Elliptic Curve}
% secp256k1
% https://crypto.stackexchange.com/questions/18965/is-secp256r1-more-secure-than-secp256k1

% \subsubsection{Asymmetric Encription <-- Needed????}



