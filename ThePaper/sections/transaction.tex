
\section{Design GSLS 2.0}

\subsection{Choosing the blockchain}
The choice of the most appropriate blockchain technologyshould take into account:

\begin{itemize}
  \item Acknowledged security: the blockchain should not contain any technical flaw and be recognized by the community as \textit{trustable}. This is important since many projects are based on a good and promising concept but often times fail on the practical implementation.
  \item Production stage: many blockchains on the market are still young and premature. This problem is probably due to the fact that investors want to go fast on the market, creating overhead and confusion to the third party services that will embraced a yet-not-ready product.
  \item Contract execution: the execution of the contracts allows to define rules (e.g. which authorities is is able to change a GlobalId and the modalities of such operations).
  \item Decentralized Control: since it is a blockchain prerogative, the technology should not be owned by a restricted group of actors and everyone must be able to join at will.
\end{itemize}

In table \ref{table:1} we analyse the most famous blockchains.

\begin{table}[h!]
\centering
\begin{tabular}{ |p{7cm}|p{2cm}|p{2cm}|p{2cm}|p{2cm}|  }
\hline
\multicolumn{5}{|c|}{Chosing the blockchain} \\
\hline
Blockchain candidate & Security & Production Stage & Contract Execution & Public \\
\hline
Bitcoin and AltCoin \cite{Nakamoto_bitcoin:a} & Yes & Yes & No\cite{Bitcoin_notTuringComplete} & Yes \\
\hline
Ethereum \cite{etherum_whitepaper} & Yes & Yes & Yes & Yes \\
\hline
BigchainDB \cite{bigchaindb_whitepaper} & No \cite{_bigchaindb_bullshit} \ & No & Yes & Yes \\
\hline
Lisk (not whitepaper available a.t.m., https://lisk.io/)& Yes & No \cite{lisk_problems} & Yes & Yes \\
\hline
IOTA \cite{iota_whitepaper}& No \cite{iota_problems} & No & Yes & Yes \\
\hline
Hyperledger Fabric \cite{martindale_fabric:_2017}& Yes & Yes & Yes & No \\
\hline
R3 Corda \cite{corda_whitepaper}& Yes & Yes & Yes & No \\
\hline
\end{tabular}
\caption{Table to compare blockchains}
\label{table:1}
\end{table}

Given this constrain, the Ethereum technology seems to be the one that most fits our necessities of replacing the previous DHT system.

\begin{notation}
  TODO: \\
  Explain why ethereum and make comaprison with bitcoin (simply: bitcoin is expensive and non optimazed for key value. Other blockchain are not safe => bigchaindb or not yet fully developed => Lisk).

  Other blockchain are not worth to be mention for lack of documentation.

  How to explain this without sources if not the persona experience??
\end{notation}
The main reasons we choose this approach are:

\begin{itemize}
  \item Consistent storage by definition. Ethereum blockchain creates different replicas of the same database in each node of the network. With such solution, nodes can enter or exit the network without compromising any social record.
  \item In case of failure, it is possible to recover the state of the database and re-execute the missing transactions.
  \item Ethereum is a distributed ledger, it is not possible to forge part of the database (at least 50\% + 1 of the nodes must be corrupted). Preventing any node to send misleading social records.
  \item One of the main features of the blockchain is the endless scalability: this permits to set up a high number of nodes (multiple GSLS servers per each OSN and the users who are willing to set up one node) without facing any delay.
  Even better, increasing the size of the network and therefore the difficulty of a rewrite attack (50\% + number of nodes).
  \item Hosting one node allows to locally access the social records stored into the GSLS storage. In the previous implementation (with DHT) it was possible that one specific social record was only available in nodes far away from the requesting resource, creating a delay in the delivery of the information. 
\end{itemize}

Given these premises, Ethereum blockchain seems to be the tool for the job: providing a consistent, distribute and scalable and environment; accessible by all the player at the same time.



\subsubsection{Serializability of the Ethereum Transaction}

Put the picture made with illustrator which is self explanatory


\begin{notation}
  Need to maintain the account key pair in order to validate the social record
\end{notation}