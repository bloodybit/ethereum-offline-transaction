\section{Introduction}
\label{S:1}

\subsection{Sonic project}

\subsection{Identity management problem in Open Source Services}

Identity management is at the moment, a main issue since it might involve and compromise the privacy of the people who have access to the Internet. 
\\
The main tech companies splitted in two, on one hand those who are prioritising the privacy (e.g. Dropbox?, Apple) and, on the other hand, those who need the data generated by the users to sustain themselves (e.g. Google, Facebook).
\\
Anyway, the security of the information has always been guaranteed since it protects the data to be widespread to the concurrent companies, ending up asking the users to sign up for many, slighty different services such as the many social networks (e.g. Facebook, Instagram, Twitter, Snapchat).
\\
Implementing the Sonic interface, it is possible to unify the management of the user data, giving the possibility to access any account information within any social network.

\begin{answer}
  Where to talk about the GSLS limitations? Here or in the sonic project above?
\end{answer}


\subsection{Blockchain}

Short problem, 
Hypothesis,
a bit of what we will do and what we expect as outcome,

The blockchain technology, born in 2009 from \textbf{Satoshi Nakamoto} [check if it is spelled correctly and place a reference to the paper], is still in its embryonal phase and its possibilities are yet not entirely explored.

The main feature of this distributed ledger is to eventually reach an unanimous agreement among the different ledgers guaranteeing the integrity of the data. 

Scalability is another key feature of the blockchain, each and every ledger act like a node in a distributed system: meaning that an application running on a server can easily access the local node with almost no latency. \textbf{<- Latency? }
\\
Given these premises, blockchain seems to be the right tool for the job: providing a consistent, distribute and scalable and environment, accessable by all the player at the same time.