\section{Introduction}
\label{S:1}

We are currently living in an era where applications are heavily data-centric and rely on information provided by users. It is the era of the \textit{Online Social Networks} (OSN) \cite{gondor_sonic:_2014} (e.g. Facebook, Twitter), where a person is analyzed in terms of interests and interaction with peers, building a network of connections and messages. As the time passes, this information converges to create a digital identity. However, this identity is bound to a particular application that has the power to influence the behavior of the person, how she interacts and what she sees. This is a well-calculated lock-in effect used to bind users to a precise service \cite{gondor_distributed_2016}. 
\\

The result of the aforementioned model is that a person tends to have segregated identities (social profiles) with replicated information in different OSN platforms. This platforms cannot communicate to each other without plugins or services. SONIC proposes an approach to overcome these problem by creating a protocol to enable communication and migration of user social profiles from one platform to another. The user profile is no more replicated and is identified by a global id that is not connected to any particular application. Therefore, the management of user data in unified, giving the possibility to access the same unique piece of information from any social network. This can, in turn, improve the privacy of the user data. 
\\

To resolve identifiers to the actual user profile location, SONIC takes advantage of a distributed system called \textit{Global Social Lookup System} (GSLS) \cite{gondor_distributed_2016}. User's \textit{GobalID} and profile location are stored in an dataset, called \textit{Social Record}, managed by the GSLS \cite{gondor_distributed_2016}.
\\

The GSLS is currently implemented with Java and is exposing APIs for creating, updating and querying the Social Records. The Social Records are stored in a Distributed Hash Table (DHT) and validated with RSA private-public key encryption. However, this open-source design bears some security issues. As an example, we describe a practicable attack: it is possible to download the source code, spawn an instance of the service and add it to the network of GSLS nodes. The malicious instance can take an outdated but signed Social Record and send it to the network, causing an override of the current entry. This can be achieved because data is checked against the P2P node's signature but neither against a timestamp nor a nonce. Therefore, it is not possible to infer if every entry is the most recent one. 
\\

To overcome this and other security issues, we have been working on a new design that substitutes the DHT with blockchain, taking advantage of its distributed consensus, Byzantine fault tolerance, encryption, and unforgeability. The designed system employs Ethereum as a blockchain implementation. The reason depends on the fact that Ethereum provides the users with sufficient storage and a Turing-complete programming language that runs on a virtual machine, called Ethereum Virtual Machine (EVM). Therefore, it is possible to use the blockchain to handle logic and execute functions. In addition, it can be used both as a validation (e.g. storing the hash of an entry) and storage system. 
\\

Each instance of the GSLS system exposes a RESTful API that holds an Ethereum node and interacts with the blockchain network. For the sake of our discussion, we will use the terms Ethereum and blockchain as synonyms.
\\

The remainder of this report is structured as follows: the next section presents an overview of related works, technical framework, and background. Section 3 and 4 provide a specification of our research questions and the description of possible designs to solve our research problem. Section 5 provides an analysis of the design in terms of security and invasiveness from the user perspective, as well as an evaluation of the performances of the implementation. In the last section we undertake a discussion on the pro and cons of the proposed solution and then we conclude the report with a description of the future work. 
