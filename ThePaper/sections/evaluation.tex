% \section{Results and Analysis}
% \label{S:5}


% \subsection{Security}

% \subsubsection{In compare with the previous GSLS}

% \begin{itemize}
%   \item Now data are sent in an encrypted way - Communication channel is secure
%   \item Data are "more distributed" not only in the GSLS network but over all the nodes
%   \item Not possible to create replication/replay attack (or whatever is called, where you basically repeat the same instruction)
%   \item Built in versioning system
%   \item Nonce problem (one node could always return a wrong nonce), change the GSLS server
% \end{itemize}

% \subsubsection{In compare with light client}
% \begin{itemize}
%   \item This is a light client, just very light, it allows only few operations (not possible to hack it with other transactions if not the one we wrote in the abi file ) <-- confirm that
%   \item All the security of a cold wallet (no 3rd party knows it), but still able to create transactions.
%   \item No need to install particular and specific blockchain nodes or to run any particular program in background. The Identity management over the Social Networks is a service widely used by the majority of the developed countries therefore must be easy to use and plug & play. [in our case you register as if a normal website -> with the same complexity for the user]
% \end{itemize}