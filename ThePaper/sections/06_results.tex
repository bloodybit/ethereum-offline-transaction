\section{Results and Analysis}
\label{S:5}


% \subsection{Security}

% \subsubsection{In compare with the previous GSLS}

% \begin{itemize}
%   \item Now data are sent in an encrypted way - Communication channel is secure
%   \item Data are "more distributed" not only in the GSLS network but over all the nodes
%   \item Not possible to create replication/replay attack (or whatever is called, where you basically repeat the same instruction)
%   \item Built in versioning system
%   \item Nonce problem (one node could always return a wrong nonce), change the GSLS server
% \end{itemize}

% \subsubsection{In compare with light client}
% \begin{itemize}
%   \item This is a light client, just very light, it allows only few operations (not possible to hack it with other transactions if not the one we wrote in the abi file ) <-- confirm that
%   \item All the security of a cold wallet (no 3rd party knows it), but still able to create transactions.
%   \item No need to install particular and specific blockchain nodes or to run any particular program in background. The Identity management over the Social Networks is a service widely used by the majority of the developed countries therefore must be easy to use and plug & play. [in our case you register as if a normal website -> with the same complexity for the user]
% \end{itemize}


In this implementation, GSLS becomes an Ethereum node which proxies the user requests to the Ethereum node itself.
The users must download a desktop application (or can build the transaction by themselves). Is recommended that this process is carried out offline: in order to avoid hijacking attack.

\subsection{Solidity data structures}
Solidity is the name of the Ethereum contract language. It is designed to optimize the efficiency and memory of the EVM and, for these reasons, it does not allow arrays with undefined length.
Therefore we map types in a key-value fashion.

One data structure is requires to map the sender's address with its own \textit{globalId} (the user unique identifier for the OSN).  To increase the read performance, another mapping has been designed: from \textit{globalId} to \textit{socialRecord} information.

This approach might be criticized because it implements one more data structure: especially considering that the storage increase exponentially with the number of nodes but, nevertheless justified by the fact of having an iterable map requires the same additional data structure by definition as in \cite{datastructure_solidity}.

\begin{notation}
  Should we add image or table with the data structures?
\end{notation}


\subsection{Write operation}
The user must create or own a \textit{key pair} (public \& private key) on the client side (See Electron desktop application in the appendix).
In order to create a new transaction, the sender must know his \textit{nonce}, the current \textit{gasPrice} and the \textit{gasLimit}. This values are available to any node, therefore the desktop application sends a get request to the GSLS along with its account address. Once all the fields are known to the sender, she signs it with her private key, encodes it via RLP and sends to the GSLS server.

Once the GSLS server receives the transaction it can optionally verify the content of the transaction (finding the public key of the sender via elliptic curve recovery) or directly forward the transaction to Ethereum node.

The transaction will there be analysed and ran in the EVM: therefore, if correct, validated, mined and broadcasted.


\subsection{Read operation}

The digital resource or physical user that wants to access the \textit{social record} given a \textit{globalId} can retrieve this information from the blockchain via a \textit{message call}.
This operation is publicly available suggesting that a proper transaction is still needed but, in this case, can be encrypted by any ethereum user (e.g. a server ad-hoc created account), thus this message call is distinguished by the keyword \textit{constant} which means that the execution of such instruction does not change the state.
In other words the situation described in \ref{ethereumEcosystem:1} $t(\sigma_t) = \sigma_{t+1}$ is such as $\sigma_t = \sigma_{t+1}$ and therefore the specific transaction $t$ can be ignored by the distributed system and be run locally without any gas cost.

Read operations are in fact free and, in an optimal configuration where the GSLS 2.0 and the Ethereum node are on the same host, fast since the information is locally available

\begin{notation}
  To do...
\end{notation}


\begin{notation}
  Append contract in the appendix
\end{notation}