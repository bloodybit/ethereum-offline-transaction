%%%%%%%%%%%%%%%%%%%%%%%%%%%%%%%%%%%%%%%%%
% Short Sectioned Assignment
% LaTeX Template
% Version 1.0 (5/5/12)
%
% This template has been downloaded from:
% http://www.LaTeXTemplates.com
%
% Original author:
% Frits Wenneker (http://www.howtotex.com)
%
% License:
% CC BY-NC-SA 3.0 (http://creativecommons.org/licenses/by-nc-sa/3.0/)
%
%%%%%%%%%%%%%%%%%%%%%%%%%%%%%%%%%%%%%%%%%

%----------------------------------------------------------------------------------------
%	PACKAGES AND OTHER DOCUMENT CONFIGURATIONS
%----------------------------------------------------------------------------------------

\documentclass[paper=a4, fontsize=11pt]{scrartcl} % A4 paper and 11pt font size

\usepackage[utf8]{inputenc}
\usepackage[T1]{fontenc} % Use 8-bit encoding that has 256 glyphs
\usepackage{fourier} % Use the Adobe Utopia font for the document - comment this line to return to the LaTeX default
\usepackage[english]{babel} % English language/hyphenation
\usepackage{amsmath,amsfonts,amsthm} % Math packages

\usepackage{lipsum} % Used for inserting dummy 'Lorem ipsum' text into the template

\usepackage{sectsty} % Allows customizing section commands
\allsectionsfont{\centering \normalfont\scshape} % Make all sections centered, the default font and small caps

\usepackage{fancyhdr} % Custom headers and footers
\pagestyle{fancyplain} % Makes all pages in the document conform to the custom headers and footers
\fancyhead{} % No page header - if you want one, create it in the same way as the footers below
\fancyfoot[L]{} % Empty left footer
\fancyfoot[C]{\thepage} % Page numbering for center footer
\fancyfoot[R]{} % Empty right footer
\renewcommand{\headrulewidth}{0pt} % Remove header underlines
\renewcommand{\footrulewidth}{0pt} % Remove footer underlines
\setlength{\headheight}{13.6pt} % Customize the height of the header

\numberwithin{equation}{section} % Number equations within sections (i.e. 1.1, 1.2, 2.1, 2.2 instead of 1, 2, 3, 4)
\numberwithin{figure}{section} % Number figures within sections (i.e. 1.1, 1.2, 2.1, 2.2 instead of 1, 2, 3, 4)
\numberwithin{table}{section} % Number tables within sections (i.e. 1.1, 1.2, 2.1, 2.2 instead of 1, 2, 3, 4)

\setlength\parindent{0pt} % Removes all indentation from paragraphs - comment this line for an assignment with lots of text

%----------------------------------------------------------------------------------------
%	TITLE SECTION
%----------------------------------------------------------------------------------------

\newcommand{\horrule}[1]{\rule{\linewidth}{#1}} % Create horizontal rule command with 1 argument of height

\title{
\normalfont \normalsize
\textsc{Kungliga Tekniska Högskolan, Research Methodologies} \\ [10pt] % Your university, school and/or department name(s)
Project Plan \\ [25pt]
\horrule{0.5pt} \\[0.4cm] % Thin top horizontal rule
\huge Title:\\Ethereum offline transaction \\ % The assignment title
\vspace{5mm}
\horrule{2pt} \\[0.5cm] % Thick bottom horizontal rule
}

\author{Riccardo Sibani \\ email \href{mailto:riccardo.sibani@gmail.com}{riccardo.sibani@gmail.com}
   \and Filippo Boiani \\ email \href{mailto:filippo.boiani2@gmail.com}{filippo.boiani2@gmail.com} } % Your name

\date{\normalsize\today} % Today's date or a custom date

\begin{document}

\maketitle % Print the title

%----------------------------------------------------------------------------------------
%	PROBLEM 1
%----------------------------------------------------------------------------------------

\section{Allocation of reponsibilities}
Riccardo Sibani is in charge of writing the first draft, composing the structure of the paper and explaining the process, demonstrating on a theoretical basis how to achieve the offline transaction.
Filippo Boiani is in charge of developing the script in order to test the presented assumptions as well as testing the performance of the suggested solution.

\section{Organization}
The project will be organized as a two-person project, building upon previously develop solution at TU Berlin. Once the theoretical process is defined and the implementation ready, there will be an evaluation work.

\section{Background}

This paper is based on a project for TU Berlin in collaboration with Deutsche Telekom. The offline solution was developed in order to give the users the possibility to update their social records (stored into the Ethereum public blockchain) without the constraint of downloading the entire blockchain node or use a third party node (which can be malicius and steal the blockchain credentials).

\section{Problem statement}


\section{Problem}

\section{Hypothesis}

\section{Purpose}

\section{Goal(s)}

\section{Tasks}

\section{Method}

\section{Milestone chart}

\section{References}


\end{document}
